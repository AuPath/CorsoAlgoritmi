% Created 2019-11-02 sab 12:40
% Intended LaTeX compiler: pdflatex
\documentclass[11pt]{article}
\usepackage[utf8]{inputenc}
\usepackage[T1]{fontenc}
\usepackage{graphicx}
\usepackage{grffile}
\usepackage{longtable}
\usepackage{wrapfig}
\usepackage{rotating}
\usepackage[normalem]{ulem}
\usepackage{amsmath}
\usepackage{textcomp}
\usepackage{amssymb}
\usepackage{capt-of}
\usepackage{hyperref}
\author{Aurelio}
\date{\today}
\title{}
\hypersetup{
 pdfauthor={Aurelio},
 pdftitle={},
 pdfkeywords={},
 pdfsubject={},
 pdfcreator={Emacs 26.3 (Org mode 9.1.9)}, 
 pdflang={English}}
\begin{document}

\tableofcontents

\section{ALGORITMI}
\label{sec:org7bd27b3}

\subsection{Programmazione dinamica}
\label{sec:orge0c66d9}

\subsubsection{Introduzione}
\label{sec:org639cb23}

L' idea è quella di non dover ripetere calcoli già fatti in precedenza, cosa che succede per 
molto di frequente in un albero di chiamate ricorsive.
Una soluzione è quella di "salvare" soluzioni a problemi intermedi in una struttura di supporto
è di controllare se un particolare calcolo è già stato fatto.
I problemi risolti dalla programmazione dinamica hanno la caratteristica di essere facilmente
risolvibili per "piccole" istanze del problema, praticamente l'analogo del caso base in una soluzione ricorsiva.
Combinando soluzioni intermedie piu semplici da risolvere si ottiene la soluzione dell' istanza completa (e più complessa)
del problema originale.


\subsubsection{Approccio al problema}
\label{sec:org6d4d4ab}

Cominciare definendo l' istanza del problema e la soluzione del problema in maniera formale.
Può essere utile, ma non è obbligatorio, definire l' equazione di ricorrenza del problema.
Anche se in seguito si utilizzerà un approccio di programmazione dinamica questo passaggio ci permette di 
individuare il caso base e lo step per ipoteticamente raggiungere la soluzione.

Una volta individuata la struttura di supporto più appropriata per il problema definire cosa
la struttura contiente ad una particolare posizione.
Generalmente contiene la soluzione fino a il/i carattere/i della sequenza/sequenze corrispondente alla
posizione della struttura di supporto. Molto importante è definire se la soluzione include o meno il carattere.


\subsubsection{Algoritmi visti in classe}
\label{sec:org30abaaf}

\begin{enumerate}
\item Longest increasing subsequence (LIS)
\label{sec:orgc8e2147}

\begin{enumerate}
\item Definizione del problema
\label{sec:orgf7e5ee8}

Avendo una sequenza \textbf{X} di lunghezza \textbf{n} con indice \textbf{I} = \{1,\ldots{},n\}, \\
una sottosequenza \textbf{Z} di \textbf{X} viene ritenuta valida se gli indici degli elementi
di \textbf{X} che la compongono rispettano un ordine strettamente crescente.

Esempio: \\
\textbf{X} = < a, b, c, d >  \textbf{Z} = < a, c , d > è valida \\
\textbf{X} = < a, b, c, d >  \textbf{Z} = < d, a, b, c > non è valida \\

Vogliamo trovare la più lunga sottosequenza dove ogni elemento è maggiore del precedente (LIS).\\
L' algoritmo visto in classe si limita a trovare la lunghezza della LIS. \\

Istanza del problema: \\
Una sequenza \textbf{X} di elementi, la lunghezza di \textbf{X} è \textbf{n} \\

Soluzione del problema: \\
\begin{itemize}
\item Se voglio la LIS: \\
Una permutazione di \textbf{Z} di \textbf{X} tale che:
\begin{enumerate}
\item Tra tutte le permutazioni possibili di \textbf{X} (insieme delle parti) \textbf{Z} risulta la sottosequenza (deve rispettare le proprietà di una sottosequenza) quello con lunghezza (cardinalità) massima.
\item Gli elementi della sottosequenza "ottima" \textbf{Z} devono essere ordinati in senso strettamente crescente.
\end{enumerate}

\item Se voglio la lunghezza della LIS: \\
La lunghezza l \(\in\) \textbf{N} della sottosequenza \textbf{Z} definita sopra.
\end{itemize}


\item Pseudocodice con spiegazione
\label{sec:org4a01437}

Come struttura di supporto viene utilizzato un array \textbf{L} di dimensione uguale alla
lunghezza \textbf{n} della sequenza \textbf{S}.
Alla posizione \textbf{i} il vettore \textbf{L} contiene la lunghezza massima della LIS fino al
carattere \textbf{i} della sequenza \textbf{S} con S[i] compreso.

Genericamente L[i] può avere valore:
\begin{itemize}
\item L[i] = 1 se i = 1\\
cioè al primo elemento della sequenza la lunghezza massima possibile è 1.
\item L[i] = 1 + MAX\{L[i]\} se 1 \(\le\) j \(\le\) i-1 AND S[j] < S[i] \\
cioè se l'elemento \textbf{i} di \textbf{S} risulta maggiore dei precedenti il valore in \textbf{L} alla posizione \textbf{i}
è uguale al valore massimo precedentemente trovato + 1, appunto perchè è stato possibile aggiungere
un nuovo elemento alla LIS fino a \textbf{i}.
\end{itemize}

Praticamente quello che l' algoritmo fa è:
\begin{enumerate}
\item imposto a 1 la lunghezza massima della LIS fino a \textbf{i} = 1.
\item scorro S e per ogni elemento controllo se è maggiore dell' elemento precedente che mi permette
di ottenere la LIS massima, questo dato lo trovo in \textbf{L}.
\item Se non trovo nessun elemento precedente L[i] sarà 1, se lo trovo sarà max + 1.
\item Dopo aver controllato ogni elemento controllo se la LIS terminante con lui ha lunghezza maggiore della
LIS massima trovata fino ad ora, se è maggiore aggiorno il valore massimo.
\end{enumerate}

Soluzione vista a lezione:
\begin{verbatim}
int LIS (S[1,...,n])
L[1] = 1;
max_tot = 1;
for i = 2 to n
  max = 0;
  for j = 1 to i-1
  if S[j] < S[i] AND L[j] > max
    max = L[J]
  L[i] = 1 + max;
  if L[i] > max_tot
    max_tot = L[i]

return max_tot
\end{verbatim}


\item Implementazione C++
\label{sec:org1e1b95f}

\begin{verbatim}
int lis(int sequence[], int size){

  // support array, at pos i will hold longest lis including i
  int l[size]; 
  int max_length = 1;
  l[0] = 1;

  for(int i = 1; i < size; ++i)
    {
      int max = 0;
      for(int j = 0; j <= i-1; ++j)
	if(sequence[i] > sequence[j] && l[j] > max)
	  max = l[j];

      l[i] = 1 + max;
      if(l[i] > max_length)
	max_length = l[i];
    }

  return max_length;
}
\end{verbatim}
\end{enumerate}

\item Longest common subsequence (LCS)
\label{sec:org136425d}
\end{enumerate}
\end{document}
